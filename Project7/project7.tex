%%%%%%%%%%%%%%%%%%%%%%%%%%%%%%%%%%%%%%%%%
% University/School Laboratory Report
% LaTeX Template
% Version 3.0 (4/2/13)
%
% This template has been downloaded from:
% http://www.LaTeXTemplates.com
%
% Original author:
% Linux and Unix Users Group at Virginia Tech Wiki
% (https://vtluug.org/wiki/Example_LaTeX_chem_lab_report)
%
% License:
% CC BY-NC-SA 3.0 (http://creativecommons.org/licenses/by-nc-sa/3.0/)
%
%%%%%%%%%%%%%%%%%%%%%%%%%%%%%%%%%%%%%%%%%

%----------------------------------------------------------------------------------------
%	PACKAGES AND DOCUMENT CONFIGURATIONS
%----------------------------------------------------------------------------------------

\documentclass{article}
\usepackage{lipsum} % Package to generate dummy text throughout this template
\usepackage[sc]{mathpazo} % Use the Palatino font
\usepackage[T1]{fontenc} % Use 8-bit encoding that has 256 glyphs
%\linespread{1.5} % Line spacing - Palatino needs more space between lines
\usepackage{microtype} % Slightly tweak font spacing for aesthetics

\usepackage[hmarginratio=1:1,top=32mm,columnsep=20pt]{geometry} % Document margins

\usepackage[hang, small,labelfont=bf,up,textfont=it,up]{caption} % Custom captions under/above floats in tables or figures
\usepackage{booktabs} % Horizontal rules in tables
\usepackage{float} % Required for tables and figures in the multi-column environment - they need to be placed in specific locations with the [H] (e.g. \begin{table}[H])

\usepackage{lettrine} % The lettrine is the first enlarged letter at the beginning of the text
\usepackage{paralist} % Used for the compactitem environment which makes bullet points with less space between them

\usepackage{fancyhdr} % Headers and footers
\pagestyle{fancy} % All pages have headers and footers
\fancyhead{} % Blank out the default header
\fancyfoot{} % Blank out the default footer
\fancyhead[C]{Operating System, Spring 2013} % Custom header text
\fancyfoot[RO]{\thepage} % Custom footer text

\usepackage{mhchem} % Package for chemical equation typesetting
\usepackage{siunitx} % Provides the \SI{}{} command for typesetting SI units

\usepackage{graphicx} % Required for the inclusion of images

\usepackage{titlesec} % Allows customization of titles
\renewcommand\thesection{\Roman{section}}
\titleformat{\section}[block]{\large\scshape\centering}{\thesection.}{1em}{} % Change the look of the section titles

%\setlength\parindent{0pt} % Removes all indentation from paragraphs

%\renewcommand{\labelenumi}{\alph{enumi}.} % Make numbering in the enumerate environment by letter rather than number (e.g. section 6)

%\usepackage{times} % Uncomment to use the Times New Roman font


%----------------------------------------------------------------------------------------
%	DOCUMENT INFORMATION
%----------------------------------------------------------------------------------------

\title{Project 7 \\ Simple Shell Interface\thanks{Designed by \LaTeX}} % Title

\author{Jiashuo \textsc{Wang} \\ 5100309436} % Author name

\date{\today} % Date for the report

\begin{document}

\maketitle % Insert the title, author and date
\thispagestyle{fancy} % All pages have headers and footers

\begin{center}
\begin{tabular}{l r}
Instructor: & Ling Gong % Instructor/supervisor
\end{tabular}
\end{center}

%\fontsize{12pt}{13pt}\selectfont

% If you wish to include an abstract, uncomment the lines below
% \begin{abstract}
% Abstract text
% \end{abstract}

%----------------------------------------------------------------------------------------
%	SECTION 1
%----------------------------------------------------------------------------------------

\section{Objective}

Write a program that can imitate a simple shell and execute some basic commands. You can see the detail in the end of chapter 3 of \emph{OPERATING SYSTEM CONCEPTS WITH JAVA(Seventh Edition)}, page 127.


% If you have more than one objective, uncomment the below:
%\begin{description}
%\item[First Objective] \hfill \\
%Objective 1 text
%\item[Second Objective] \hfill \\
%Objective 2 text
%\end{description}

%----------------------------------------------------------------------------------------
%	SECTION 2
%----------------------------------------------------------------------------------------

\section{Algorithm}

\subsection{Simple Shell}
The key is to create an external process to let the Linux operating system to help me execute most of the command based on its command document in the Linux. So this program can only be executed in the OS like Linux, not Windows because the command in Linux is in Command documents. This kind of commands is like \emph{ls}, \emph{cat}, \emph{pwd}.

Some command, however, is not in the system, which means we have to implement it by ourselves. These are like \emph{cd}, \emph{history}, \emph{exit}.
\begin{enumerate}
  \item ewfdww
  \begin{enumerate}
    \item sasa
    \begin{enumerate}
      \item dsds
      \begin{enumerate}
        \item ds
        \item fe
        \item fe
      \end{enumerate}
      \item dsd
      \item as
    \end{enumerate}
    \item sas
    \item dsd
  \end{enumerate}
  \item wdsd
  \item dssds
\end{enumerate}

\subsection{\emph{cd} Implementation}
To implement \emph{cd}, we have to take several situation into account. For example:
\begin{equation}
cd\ \ ..
\end{equation}
\begin{equation}
cd\ \ /usr/bin
\end{equation}
\begin{equation}
cd\ \ src
\end{equation}

So first, we have to get the current directory with \emph{getProperty()}. For (1), we can get the directory by wiping out the last file. But if the current directory has already been the root one, the directory has to be "/". For (2)(3), we have to judge whether the directory is legal by using \emph{isDirectory{}}. Then for (2), we can replace the directory by the content after the command \emph{cd}. For (3), we can add the content after the original directory.

\subsection{\emph{history} Implementation}
The \emph{history} function is designed to imitate that in Linux \emph{Terminal}.

Every time the user input something, the \emph{ArrayList} will help to store them if the input command is not the same as the last one. It is a dynamic array. So when you input \emph{history} command, the historical commands will print.

When you input the command \emph{!!}, the last command will be executed automatically. Mean while, if you input not only \emph{!!}, but also something else, the rest one will be add to the historical command automatically. For example, if the last command is \emph{ls},and now you input \emph{!! -l}, the command to be executed will be \emph{ls -l}.

When you input the command \emph{!} and a number, that means you want to execute the historical command with the index of that number. And that command will be executed automatically.

\subsection{More details}
If the command one input is not Linux command, an \emph{IOException} will be threw. And the shell should tell the user the error message and continue the shell. And when the command is \emph{!!} or like \emph{!4}, the limit of the history will be considered.

And if the user inputs the \emph{exit}, the endless loop should be broken to terminate the program.

%----------------------------------------------------------------------------------------
%	SECTION 3
%----------------------------------------------------------------------------------------

\section{Results and Conclusions}
\subsection{Environment}
\begin{compactitem}
\item Ubuntu 12.10
\item Eclipse jee juno SR2 for Linux
\end{compactitem}
\subsection{Screenshots of the result}
Use JVM to compile and execute the program in Figure 1.
\begin{figure}[h]
\centering
\includegraphics[height=9cm]{1.eps}
\includegraphics[height=7cm]{2.eps}
\caption{Screenshots of Matrix Multiplication}
\end{figure}

\subsection{Thoughts}
It is amazing to make a program just like an OS, which is quite interesting.

%----------------------------------------------------------------------------------------


\end{document}

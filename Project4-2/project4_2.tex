%%%%%%%%%%%%%%%%%%%%%%%%%%%%%%%%%%%%%%%%%
% University/School Laboratory Report
% LaTeX Template
% Version 3.0 (4/2/13)
%
% This template has been downloaded from:
% http://www.LaTeXTemplates.com
%
% Original author:
% Linux and Unix Users Group at Virginia Tech Wiki
% (https://vtluug.org/wiki/Example_LaTeX_chem_lab_report)
%
% License:
% CC BY-NC-SA 3.0 (http://creativecommons.org/licenses/by-nc-sa/3.0/)
%
%%%%%%%%%%%%%%%%%%%%%%%%%%%%%%%%%%%%%%%%%

%----------------------------------------------------------------------------------------
%	PACKAGES AND DOCUMENT CONFIGURATIONS
%----------------------------------------------------------------------------------------

\documentclass{article}
\usepackage{lipsum} % Package to generate dummy text throughout this template
\usepackage[sc]{mathpazo} % Use the Palatino font
\usepackage[T1]{fontenc} % Use 8-bit encoding that has 256 glyphs
%\linespread{1.5} % Line spacing - Palatino needs more space between lines
\usepackage{microtype} % Slightly tweak font spacing for aesthetics

\usepackage[hmarginratio=1:1,top=32mm,columnsep=20pt]{geometry} % Document margins

\usepackage[hang, small,labelfont=bf,up,textfont=it,up]{caption} % Custom captions under/above floats in tables or figures
\usepackage{booktabs} % Horizontal rules in tables
\usepackage{float} % Required for tables and figures in the multi-column environment - they need to be placed in specific locations with the [H] (e.g. \begin{table}[H])

\usepackage{lettrine} % The lettrine is the first enlarged letter at the beginning of the text
\usepackage{paralist} % Used for the compactitem environment which makes bullet points with less space between them

\usepackage{fancyhdr} % Headers and footers
\pagestyle{fancy} % All pages have headers and footers
\fancyhead{} % Blank out the default header
\fancyfoot{} % Blank out the default footer
\fancyhead[C]{Operating System, Spring 2013} % Custom header text
\fancyfoot[RO]{\thepage} % Custom footer text

\usepackage{mhchem} % Package for chemical equation typesetting
\usepackage{siunitx} % Provides the \SI{}{} command for typesetting SI units

\usepackage{graphicx} % Required for the inclusion of images

\usepackage{titlesec} % Allows customization of titles
\renewcommand\thesection{\Roman{section}}
\titleformat{\section}[block]{\large\scshape\centering}{\thesection.}{1em}{} % Change the look of the section titles

%\setlength\parindent{0pt} % Removes all indentation from paragraphs

%\renewcommand{\labelenumi}{\alph{enumi}.} % Make numbering in the enumerate environment by letter rather than number (e.g. section 6)

%\usepackage{times} % Uncomment to use the Times New Roman font


%----------------------------------------------------------------------------------------
%	DOCUMENT INFORMATION
%----------------------------------------------------------------------------------------

\title{Project 4-2 \\ I/O and NetWork Downloading\thanks{Designed by \LaTeX}} % Title

\author{Jiashuo \textsc{Wang} \\ 5100309436} % Author name

\date{\today} % Date for the report

\begin{document}

\maketitle % Insert the title, author and date
\thispagestyle{fancy} % All pages have headers and footers

\begin{center}
\begin{tabular}{l r}
Instructor: & Ling Gong % Instructor/supervisor
\end{tabular}
\end{center}

%\fontsize{12pt}{13pt}\selectfont

% If you wish to include an abstract, uncomment the lines below
% \begin{abstract}
% Abstract text
% \end{abstract}

%----------------------------------------------------------------------------------------
%	SECTION 1
%----------------------------------------------------------------------------------------

\section{Objective}

Create a class $WGet$ that takes a URL and a filename on the command line, and attempts to retrieve the data at that URL into the specified file. For example, "java WGet http://www.cs.columbia.edu/index.html myfile.html", then you are retrieving the file at CS website and store it into your own file $myfile.html$.
It should fail if the filename you are trying to write to is already existing, and fail with an informative message if something goes wrong (i.e. don't just dump a stack trace).

You can assume that the data is binary, i.e. don't worry about whether to use a $Reader$, just use an $InputStream$. You can also assume the file is text only. Just notify it in your code.

% If you have more than one objective, uncomment the below:
%\begin{description}
%\item[First Objective] \hfill \\
%Objective 1 text
%\item[Second Objective] \hfill \\
%Objective 2 text
%\end{description}

%----------------------------------------------------------------------------------------
%	SECTION 2
%----------------------------------------------------------------------------------------

\section{Algorithm}

\subsection{Input Stream from URL}
In order to retrieve the data at that URL into the specified file, it is necessary to get the data from the URL. Then class $URL$ and $URLConnection$ is used here to found a connection to the web. Then with the method $getInputStream()$, we can download the service data to get ready to the file HTML.

\subsection{Output Stream to HTML}
To put the data into the file, the file I/O stream is useful. Class $FileOutputStream$ can do something to create output stream and create the file with the file type of HTML.

\subsection{More details}
\begin{compactitem}
\item With the class $File$ and its method $exists()$, we can easily check if there is already a file named that. If there is, then it will fail to write.
\item During the operation, several errors will happen. So $try-catch$ statement is needed to catch the errors.
\item Generally if using $System.out.println()$, the output will print on the screen. So we have to use $System.setOut(out)$ to redirect the system output to the file.
\end{compactitem}

%----------------------------------------------------------------------------------------
%	SECTION 3
%----------------------------------------------------------------------------------------

\section{Results and Conclusions}
\subsection{Environment}
\begin{compactitem}
\item Windows 8
\item NetBeans IDE 7.3
\end{compactitem}

\subsection{Screenshots of the result}
Use Command Line to compile and execute the program in Figure 1. Then we can see the interface is the same shown in Figure 2 and Figure 3.

\subsection{Thoughts}
Web programming with Java is useful. In this lab, I consult lots of information on the Internet about it. Now I have some knowledge about web programming with Java. It's very interesting!

\begin{figure}[h]
\centering
\includegraphics[height=2cm]{1.eps}
\caption{Screenshots of Execution Interface}
\end{figure}
\begin{figure}[h]
\centering
\includegraphics[height=9cm]{2.eps}
\caption{Screenshots of the Original Website}
\end{figure}
\begin{figure}[h]
\centering
\includegraphics[height=9cm]{3.eps}
\caption{Screenshots of the HTML File}
\end{figure}
%----------------------------------------------------------------------------------------


\end{document}

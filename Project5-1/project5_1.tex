%%%%%%%%%%%%%%%%%%%%%%%%%%%%%%%%%%%%%%%%%
% University/School Laboratory Report
% LaTeX Template
% Version 3.0 (4/2/13)
%
% This template has been downloaded from:
% http://www.LaTeXTemplates.com
%
% Original author:
% Linux and Unix Users Group at Virginia Tech Wiki
% (https://vtluug.org/wiki/Example_LaTeX_chem_lab_report)
%
% License:
% CC BY-NC-SA 3.0 (http://creativecommons.org/licenses/by-nc-sa/3.0/)
%
%%%%%%%%%%%%%%%%%%%%%%%%%%%%%%%%%%%%%%%%%

%----------------------------------------------------------------------------------------
%	PACKAGES AND DOCUMENT CONFIGURATIONS
%----------------------------------------------------------------------------------------

\documentclass{article}
\usepackage{lipsum} % Package to generate dummy text throughout this template
\usepackage[sc]{mathpazo} % Use the Palatino font
\usepackage[T1]{fontenc} % Use 8-bit encoding that has 256 glyphs
%\linespread{1.5} % Line spacing - Palatino needs more space between lines
\usepackage{microtype} % Slightly tweak font spacing for aesthetics

\usepackage[hmarginratio=1:1,top=32mm,columnsep=20pt]{geometry} % Document margins

\usepackage[hang, small,labelfont=bf,up,textfont=it,up]{caption} % Custom captions under/above floats in tables or figures
\usepackage{booktabs} % Horizontal rules in tables
\usepackage{float} % Required for tables and figures in the multi-column environment - they need to be placed in specific locations with the [H] (e.g. \begin{table}[H])

\usepackage{lettrine} % The lettrine is the first enlarged letter at the beginning of the text
\usepackage{paralist} % Used for the compactitem environment which makes bullet points with less space between them

\usepackage{fancyhdr} % Headers and footers
\pagestyle{fancy} % All pages have headers and footers
\fancyhead{} % Blank out the default header
\fancyfoot{} % Blank out the default footer
\fancyhead[C]{Operating System, Spring 2013} % Custom header text
\fancyfoot[RO]{\thepage} % Custom footer text

\usepackage{mhchem} % Package for chemical equation typesetting
\usepackage{siunitx} % Provides the \SI{}{} command for typesetting SI units

\usepackage{graphicx} % Required for the inclusion of images

\usepackage{titlesec} % Allows customization of titles
\renewcommand\thesection{\Roman{section}}
\titleformat{\section}[block]{\large\scshape\centering}{\thesection.}{1em}{} % Change the look of the section titles

%\setlength\parindent{0pt} % Removes all indentation from paragraphs

%\renewcommand{\labelenumi}{\alph{enumi}.} % Make numbering in the enumerate environment by letter rather than number (e.g. section 6)

%\usepackage{times} % Uncomment to use the Times New Roman font


%----------------------------------------------------------------------------------------
%	DOCUMENT INFORMATION
%----------------------------------------------------------------------------------------

\title{Project 5-1 \\ Multi-Thread Programming\thanks{Designed by \LaTeX}} % Title

\author{Jiashuo \textsc{Wang} \\ 5100309436} % Author name

\date{\today} % Date for the report

\begin{document}

\maketitle % Insert the title, author and date
\thispagestyle{fancy} % All pages have headers and footers

\begin{center}
\begin{tabular}{l r}
Instructor: & Ling Gong % Instructor/supervisor
\end{tabular}
\end{center}

%\fontsize{12pt}{13pt}\selectfont

% If you wish to include an abstract, uncomment the lines below
% \begin{abstract}
% Abstract text
% \end{abstract}

%----------------------------------------------------------------------------------------
%	SECTION 1
%----------------------------------------------------------------------------------------

\section{Objective}

Create a class named $Countdown$ that uses an explicit $Thread$ (don't use the $Timer$ class for this homework) to write a countdown to the console, one number per second. You can either extend $Thread$ or implement $Runnable$. Your solution is free to implement other classes, but main() should be in Countdown.

The program will not actually exit until the user hits the enter key. If the user hits the enter key before the countdown has finished, the program should print interrupted and stop immediately. See transcripts below:

java Countdown

10

9

8

7

6

5

4

3

2

1

0

user hits the Enter key

finished

java Countdown

10

9

8

7

6

5

4

user hits the Enter key

interrupted

finished

\textbf{Notes:}
\begin{compactitem}
\item The program should exit on exceptions
\item Don't use $System.exit()$ to stop your program when the user hits Enter (although you can put it in your Exception handlers)
\item the $readLine()$ and $sleep()$ methods will be useful
\item If you have any questions, send me an e-mail. You can also check out the Java tutorial on Threads.
\end{compactitem}

\textbf{Extra clarification/hint:}

Since we are not using $System.exit()$, then a stoppable thread should check a boolean variable each time it goes around its loop to see if it should continue.

To stop the thread, one just sets the variable, and the loop will stop itself. Note that the thing stopping the thread will need a reference to the stoppable thread object. Also, the stoppable thread object must provide some public method for setting the value of the "should-I-continue" variable.


% If you have more than one objective, uncomment the below:
%\begin{description}
%\item[First Objective] \hfill \\
%Objective 1 text
%\item[Second Objective] \hfill \\
%Objective 2 text
%\end{description}

%----------------------------------------------------------------------------------------
%	SECTION 2
%----------------------------------------------------------------------------------------

\section{Algorithm}

\subsection{Create Another Thread}
In general, there is only one thread in a java process, which is the one that contains the $main$ method. In order to create another thread, you have to either extend $Thread$ or implement $Runnable$ when defining another class.
\begin{compactitem}
\item \textbf{Extend Thread}

If extending $java.lang.Thread$, I have to override $run()$ method. I can simply use $start()$ method to start a thread. And when a thread is started, $JVM$ will execute $run()$ automatically. However, if extending $Thread$, the class cannot extend other class, which is not extensible.

\item \textbf{Implement Runnable}

If implementing $java.lang.Runnable$, I also have to override $run()$ method. But in order to start it, it seems a little complicated. Firstly, I need to instantiate the $Thread$ with the constructor $Thread(Runnable target)$. Then I can just use the $start()$ method of $Thread$ to start the thread.

\end{compactitem}

In my code, I extend $Thread$ to create a new thread.

\subsection{Count Down}

By using $sleep(millisecond)$, I can easily get the interval of 1s. Then simply using $while$ loop to count down is an easy way to finish the work.

\subsection{End the Thread}

If a variable is marked with $volatile$, this variable can be shared with the last modified value. After the Enter button is pressed, thread will be interrupted into the $InterruptedException$ and the $volatile$ variable "should-I-continue" will be changed, which leads to the stop of the counter and the exit of the thread.

\subsection{More details}

Use $wait()$ to pend the thread to hand over the CPU to make CPU more efficient. And $wait()$ method should be contained in the $synchronized()$ method. 

%----------------------------------------------------------------------------------------
%	SECTION 3
%----------------------------------------------------------------------------------------

\section{Results and Conclusions}
\subsection{Environment}
\begin{compactitem}
\item Windows 8
\item NetBeans IDE 7.3
\end{compactitem}
\subsection{Screenshots of the result}
Use JVM to compile and execute the program in Figure 1 and Figure 2.

\subsection{Thoughts}
Multi-thread is useful and Java is a good way to implement it.
\begin{figure}[h]
\centering
\includegraphics[height=4cm]{1.eps}
\caption{Screenshots of Multi-Thread Programming(1)}
\end{figure}
\begin{figure}[h]
\centering
\includegraphics[height=6cm]{2.eps}
\caption{Screenshots of Multi-Thread Programming(2)}
\end{figure}
%----------------------------------------------------------------------------------------


\end{document}

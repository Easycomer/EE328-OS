%%%%%%%%%%%%%%%%%%%%%%%%%%%%%%%%%%%%%%%%%
% University/School Laboratory Report
% LaTeX Template
% Version 3.0 (4/2/13)
%
% This template has been downloaded from:
% http://www.LaTeXTemplates.com
%
% Original author:
% Linux and Unix Users Group at Virginia Tech Wiki
% (https://vtluug.org/wiki/Example_LaTeX_chem_lab_report)
%
% License:
% CC BY-NC-SA 3.0 (http://creativecommons.org/licenses/by-nc-sa/3.0/)
%
%%%%%%%%%%%%%%%%%%%%%%%%%%%%%%%%%%%%%%%%%

%----------------------------------------------------------------------------------------
%	PACKAGES AND DOCUMENT CONFIGURATIONS
%----------------------------------------------------------------------------------------

\documentclass{article}
\usepackage{lipsum} % Package to generate dummy text throughout this template
\usepackage[sc]{mathpazo} % Use the Palatino font
\usepackage[T1]{fontenc} % Use 8-bit encoding that has 256 glyphs
%\linespread{1.5} % Line spacing - Palatino needs more space between lines
\usepackage{microtype} % Slightly tweak font spacing for aesthetics

\usepackage[hmarginratio=1:1,top=32mm,columnsep=20pt]{geometry} % Document margins

\usepackage[hang, small,labelfont=bf,up,textfont=it,up]{caption} % Custom captions under/above floats in tables or figures
\usepackage{booktabs} % Horizontal rules in tables
\usepackage{float} % Required for tables and figures in the multi-column environment - they need to be placed in specific locations with the [H] (e.g. \begin{table}[H])

\usepackage{lettrine} % The lettrine is the first enlarged letter at the beginning of the text
\usepackage{paralist} % Used for the compactitem environment which makes bullet points with less space between them

\usepackage{fancyhdr} % Headers and footers
\pagestyle{fancy} % All pages have headers and footers
\fancyhead{} % Blank out the default header
\fancyfoot{} % Blank out the default footer
\fancyhead[C]{Operating System, Spring 2013} % Custom header text
\fancyfoot[RO]{\thepage} % Custom footer text

\usepackage{mhchem} % Package for chemical equation typesetting
\usepackage{siunitx} % Provides the \SI{}{} command for typesetting SI units

\usepackage{graphicx} % Required for the inclusion of images

\usepackage{titlesec} % Allows customization of titles
\renewcommand\thesection{\Roman{section}}
\titleformat{\section}[block]{\large\scshape\centering}{\thesection.}{1em}{} % Change the look of the section titles

%\setlength\parindent{0pt} % Removes all indentation from paragraphs

%\renewcommand{\labelenumi}{\alph{enumi}.} % Make numbering in the enumerate environment by letter rather than number (e.g. section 6)

%\usepackage{times} % Uncomment to use the Times New Roman font


%----------------------------------------------------------------------------------------
%	DOCUMENT INFORMATION
%----------------------------------------------------------------------------------------

\title{Project 1-2 \\ Date Calculation\thanks{Designed by \LaTeX}} % Title

\author{Jiashuo \textsc{Wang} \\ 5100309436} % Author name

\date{\today} % Date for the report

\begin{document}

\maketitle % Insert the title, author and date
\thispagestyle{fancy} % All pages have headers and footers

\begin{center}
\begin{tabular}{l r}
Instructor: & Ling Gong % Instructor/supervisor
\end{tabular}
\end{center}

%\fontsize{12pt}{13pt}\selectfont

% If you wish to include an abstract, uncomment the lines below
% \begin{abstract}
% Abstract text
% \end{abstract}

%----------------------------------------------------------------------------------------
%	SECTION 1
%----------------------------------------------------------------------------------------

\section{Objective}

This problem is much more flexible, any reasonable implementation is acceptable. Please implement a class called MyDate which can represent a date. The class should have at least three int fields: year, month, day, and several methods to handle them.  Please make sure the values set to these fields are all valid. Pay attention to the special cases: some month can only 28/29 days, and some month can have 31 days.

Other the basic methods, there are two additional required methods:

\begin{compactitem}
\item \textbf{laterThan(MyDate date)}

This method decides whether I am a later date than the passed-in parameter date, and return true or false. For example, if A is a MyDate object representing date 9/15/2004, and B representing date 9/22/2004, A.laterThan(B) should return false, while B.laterThan(A) should return true, and A.laterThan(A) should also return false.
\item \textbf{dayDifference(MyDate date1, MyDate date2)}

This method should be a static method. It takes two dates and returns an integer indicating how many days are there between them. For example, the days between 9/15/2004 and 9/22/2004 is 7.  date2 shouldn't have to be larger (later) than date1, and you need to judge it yourself.
    Hint: you may try use previous method inside this method.
\end{compactitem}

To test the code, you can write you own main() function, or any other Class you want to. You can do console input, or hard-code the date creation inside your code is also fine here. The purpose of this assignment is just let you get familiar with class definition and manipulate with objects.

% If you have more than one objective, uncomment the below:
%\begin{description}
%\item[First Objective] \hfill \\
%Objective 1 text
%\item[Second Objective] \hfill \\
%Objective 2 text
%\end{description}

%----------------------------------------------------------------------------------------
%	SECTION 2
%----------------------------------------------------------------------------------------

\section{Algorithm}

The two methods are the most important problem to be solve in this project.
\subsection{laterThan(MyDate date)}
This method is much easier to implement. Simply comparing the year, month and day in order is enough to judge which is the later one.
\begin{figure}[h]
\centering
\includegraphics[height=12cm]{1.eps}
\caption{Screenshots of Date Calculation}
\end{figure}
\subsection{dayDifference(MyDate date1, MyDate date2)}
In the class of last Friday evening, we learned a new way to solve this kind of problem, which is transport the date into the form of Julian Day and then just making a subtraction is OK.

After class, I thought finding a new way is much more interesting. So I tried to search the Internet to find a new way to do so. Then I found the Java class \textbf{Calendar}, which is also in the java.util. So in my code, I do this job in the following steps:
\begin{compactitem}
\item Turn the format of date into the format of $Calendar$;
\item Use the For statement to traverse the days between the two dates and the variance $daysBetween$ will add by 1;
\item After the traversing, we will get the days between the two dates.
\end{compactitem}

Before doing these, it is necessary, as mentioned in the requirement, to judge if the end date is later than the start date, which is easier because the first method will help to do so.

\subsection{More details}
Before using the date, we have to judge if the format of date is valid to be used. So we make a method $setDate$ first to help us check it.

First of all, I use one of the method, $String s1[] = s.split("/")$, of the class $String$ to split the date into three part, month, day, year. If there are not 3 parts judged by $if(s1.length != 3)$, you have to input again. Then I use the method, $s1[0].matches("[0-9]{1,2}")$, which was shown by our TA last Friday, to check the format for the first time. For further check, I use some If statements to judge if it is reasonable. If not, users have to input the date again.

%----------------------------------------------------------------------------------------
%	SECTION 3
%----------------------------------------------------------------------------------------

\section{Results and Conclusions}
\subsection{Environment}
\begin{compactitem}
\item Windows 8
\item NetBeans IDE 7.3
\end{compactitem}
\subsection{Screenshots of the result}
Use Command Line to compile and execute the program in Figure 1.

\subsection{Thoughts}
Finding a new way to solve an old problem is interesting!

%----------------------------------------------------------------------------------------


\end{document}

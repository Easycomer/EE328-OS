%%%%%%%%%%%%%%%%%%%%%%%%%%%%%%%%%%%%%%%%%
% University/School Laboratory Report
% LaTeX Template
% Version 3.0 (4/2/13)
%
% This template has been downloaded from:
% http://www.LaTeXTemplates.com
%
% Original author:
% Linux and Unix Users Group at Virginia Tech Wiki
% (https://vtluug.org/wiki/Example_LaTeX_chem_lab_report)
%
% License:
% CC BY-NC-SA 3.0 (http://creativecommons.org/licenses/by-nc-sa/3.0/)
%
%%%%%%%%%%%%%%%%%%%%%%%%%%%%%%%%%%%%%%%%%

%----------------------------------------------------------------------------------------
%	PACKAGES AND DOCUMENT CONFIGURATIONS
%----------------------------------------------------------------------------------------

\documentclass{article}
\usepackage{lipsum} % Package to generate dummy text throughout this template
\usepackage[sc]{mathpazo} % Use the Palatino font
\usepackage[T1]{fontenc} % Use 8-bit encoding that has 256 glyphs
%\linespread{1.5} % Line spacing - Palatino needs more space between lines
\usepackage{microtype} % Slightly tweak font spacing for aesthetics

\usepackage[hmarginratio=1:1,top=32mm,columnsep=20pt]{geometry} % Document margins

\usepackage[hang, small,labelfont=bf,up,textfont=it,up]{caption} % Custom captions under/above floats in tables or figures
\usepackage{booktabs} % Horizontal rules in tables
\usepackage{float} % Required for tables and figures in the multi-column environment - they need to be placed in specific locations with the [H] (e.g. \begin{table}[H])

\usepackage{lettrine} % The lettrine is the first enlarged letter at the beginning of the text
\usepackage{paralist} % Used for the compactitem environment which makes bullet points with less space between them

\usepackage{fancyhdr} % Headers and footers
\pagestyle{fancy} % All pages have headers and footers
\fancyhead{} % Blank out the default header
\fancyfoot{} % Blank out the default footer
\fancyhead[C]{Operating System, Spring 2013} % Custom header text
\fancyfoot[RO]{\thepage} % Custom footer text

\usepackage{mhchem} % Package for chemical equation typesetting
\usepackage{siunitx} % Provides the \SI{}{} command for typesetting SI units

\usepackage{graphicx} % Required for the inclusion of images

\usepackage{titlesec} % Allows customization of titles
\renewcommand\thesection{\Roman{section}}
\titleformat{\section}[block]{\large\scshape\centering}{\thesection.}{1em}{} % Change the look of the section titles

%\setlength\parindent{0pt} % Removes all indentation from paragraphs

%\renewcommand{\labelenumi}{\alph{enumi}.} % Make numbering in the enumerate environment by letter rather than number (e.g. section 6)

%\usepackage{times} % Uncomment to use the Times New Roman font


%----------------------------------------------------------------------------------------
%	DOCUMENT INFORMATION
%----------------------------------------------------------------------------------------

\title{Project 5-2 \\ GUI Programming\thanks{Designed by \LaTeX}} % Title

\author{Jiashuo \textsc{Wang} \\ 5100309436} % Author name

\date{\today} % Date for the report

\begin{document}

\maketitle % Insert the title, author and date
\thispagestyle{fancy} % All pages have headers and footers

\begin{center}
\begin{tabular}{l r}
Instructor: & Ling Gong % Instructor/supervisor
\end{tabular}
\end{center}

%\fontsize{12pt}{13pt}\selectfont

% If you wish to include an abstract, uncomment the lines below
% \begin{abstract}
% Abstract text
% \end{abstract}

%----------------------------------------------------------------------------------------
%	SECTION 1
%----------------------------------------------------------------------------------------

\section{Objective}

Write a program that displays the Olympic rings. Make the rings colored in the Olympic colors. If you like to make it fancier, you can add "USA" on top the rings (not required).

You are not required to draw exactly the same as in the pictures. Anything looks similar is fine. And the color in the intersection part of the rings is not strictly required.


% If you have more than one objective, uncomment the below:
%\begin{description}
%\item[First Objective] \hfill \\
%Objective 1 text
%\item[Second Objective] \hfill \\
%Objective 2 text
%\end{description}

%----------------------------------------------------------------------------------------
%	SECTION 2
%----------------------------------------------------------------------------------------

\section{Algorithm}

\subsection{Draw the Picture}
By extending $JFrame$, we are free to use some draw methods to draw. When $setVisible(true)$ is executed, $paint(Graphics\  g)$ will be executed automatically. In this method, $drawOval()$ is used to draw the circle with the coordinates of the left upper corner. Meanwhile, $setColor()$ is used to set the color of the circles. And $drawString()$ is used to print strings.

The distance among the three circles in the first row should be less than the diameter. And the centers of the nearest three circles should almost make an equilateral triangle. Then we will get the coordinates of each circles.

\subsection{More details}

To overstrike the circle, $Graphics2D$ is used to make it free  to use $setStroke()$ method. Besides, $setFont()$ is used to set the font type and font size of Strings.

%----------------------------------------------------------------------------------------
%	SECTION 3
%----------------------------------------------------------------------------------------

\section{Results and Conclusions}
\subsection{Environment}
\begin{compactitem}
\item Windows 8
\item NetBeans IDE 7.3
\end{compactitem}
\subsection{Screenshots of the result}
Use JVM to compile and execute the program in Figure 1.

\subsection{Thoughts}
With Java, GUI programming is becoming much easier.
\begin{figure}[h]
\centering
\includegraphics[height=8cm]{1.eps}
\caption{Screenshots of GUI Programming}
\end{figure}
%----------------------------------------------------------------------------------------


\end{document}

%%%%%%%%%%%%%%%%%%%%%%%%%%%%%%%%%%%%%%%%%
% University/School Laboratory Report
% LaTeX Template
% Version 3.0 (4/2/13)
%
% This template has been downloaded from:
% http://www.LaTeXTemplates.com
%
% Original author:
% Linux and Unix Users Group at Virginia Tech Wiki
% (https://vtluug.org/wiki/Example_LaTeX_chem_lab_report)
%
% License:
% CC BY-NC-SA 3.0 (http://creativecommons.org/licenses/by-nc-sa/3.0/)
%
%%%%%%%%%%%%%%%%%%%%%%%%%%%%%%%%%%%%%%%%%

%----------------------------------------------------------------------------------------
%	PACKAGES AND DOCUMENT CONFIGURATIONS
%----------------------------------------------------------------------------------------

\documentclass{article}
\usepackage{lipsum} % Package to generate dummy text throughout this template
\usepackage[sc]{mathpazo} % Use the Palatino font
\usepackage[T1]{fontenc} % Use 8-bit encoding that has 256 glyphs
%\linespread{1.5} % Line spacing - Palatino needs more space between lines
\usepackage{microtype} % Slightly tweak font spacing for aesthetics

\usepackage[hmarginratio=1:1,top=32mm,columnsep=20pt]{geometry} % Document margins

\usepackage[hang, small,labelfont=bf,up,textfont=it,up]{caption} % Custom captions under/above floats in tables or figures
\usepackage{booktabs} % Horizontal rules in tables
\usepackage{float} % Required for tables and figures in the multi-column environment - they need to be placed in specific locations with the [H] (e.g. \begin{table}[H])

\usepackage{lettrine} % The lettrine is the first enlarged letter at the beginning of the text
\usepackage{paralist} % Used for the compactitem environment which makes bullet points with less space between them

\usepackage{fancyhdr} % Headers and footers
\pagestyle{fancy} % All pages have headers and footers
\fancyhead{} % Blank out the default header
\fancyfoot{} % Blank out the default footer
\fancyhead[C]{Operating System, Spring 2013} % Custom header text
\fancyfoot[RO]{\thepage} % Custom footer text

\usepackage{mhchem} % Package for chemical equation typesetting
\usepackage{siunitx} % Provides the \SI{}{} command for typesetting SI units

\usepackage{graphicx} % Required for the inclusion of images

\usepackage{titlesec} % Allows customization of titles
\renewcommand\thesection{\Roman{section}}
\titleformat{\section}[block]{\large\scshape\centering}{\thesection.}{1em}{} % Change the look of the section titles

%\setlength\parindent{0pt} % Removes all indentation from paragraphs

%\renewcommand{\labelenumi}{\alph{enumi}.} % Make numbering in the enumerate environment by letter rather than number (e.g. section 6)

%\usepackage{times} % Uncomment to use the Times New Roman font


%----------------------------------------------------------------------------------------
%	DOCUMENT INFORMATION
%----------------------------------------------------------------------------------------

\title{Project 1-1 \\ Snake-Like Increment Table\thanks{Designed by \LaTeX}} % Title

\author{Jiashuo \textsc{Wang} \\ 5100309436} % Author name

\date{\today} % Date for the report

\begin{document}

\maketitle % Insert the title, author and date
\thispagestyle{fancy} % All pages have headers and footers

\begin{center}
\begin{tabular}{l r}
Instructor: & Ling Gong % Instructor/supervisor
\end{tabular}
\end{center}

%\fontsize{12pt}{13pt}\selectfont

% If you wish to include an abstract, uncomment the lines below
% \begin{abstract}
% Abstract text
% \end{abstract}

%----------------------------------------------------------------------------------------
%	SECTION 1
%----------------------------------------------------------------------------------------

\section{Objective}

This program will create small, snake-like increment tables, up to 9x9 in size. It starts from 1, and the value keep increasing by one. For the first row, it increase from left to right; for the second row, it increase from right to left; the third row, from left to right, ... , etc.  Just like a snake.

\subsection{Sample}
>java SnakeTable\\
\begin{table}[h]
%\centering
%\caption{Reasonable Distribution of Control Channels and Voice Channels in Each Cell}
\begin{tabular}{c c c c c c c c c }
1   &    2    &    3     &    4     &  5   &      6  &      7     &   8    &    9\\
        18    &  17  &    16  &    15  &    14 &     13&      12     & 11  &    10\\
        19 &     20  &    21    &  22    &  23  &    24    &  25  &    26 &     27\\
        36&      35   &   34   &   33     & 32  &    31  &    30  &    29  &    28\\
        37  &    38   &   39   &   40    &  41   &   42  &    43  &    44 &     45\\
        54 &     53    &  52   &   51  &    50  &    49  &    48   &   47   &   46\\
        55    &  56  &    57     & 58  &    59   &   60   &   61   &   62   &   63\\
        72    &  71   &   70    &  69  &    68    &  67   &   66   &   65  &    64\\
        73    &  74   &   75    &  76   &   77   &   78   &   79   &   80   &   81\\
\end{tabular}
\end{table}
Enter table size 1-9, 0 to exit: 9\\
\begin{table}[h]
%\centering
%\caption{Reasonable Distribution of Control Channels and Voice Channels in Each Cell}
\begin{tabular}{c c c }
1	& 2&	 3\\
	 6	& 5	& 4\\
	 7	& 8	& 9\\
\end{tabular}
\end{table}
Enter table size 1-9, 0 to exit: 3\\
\begin{table}[h]
%\centering
%\caption{Reasonable Distribution of Control Channels and Voice Channels in Each Cell}
\begin{tabular}{c}
1\\
\end{tabular}
\end{table}
Enter table size 1-9, 0 to exit: 1\\
Enter table size 1-9, 0 to exit: 746\\
please enter a number in the range 0-9\\
Enter table size 1-9, 0 to exit: -231\\
please enter a number in the range 0-9\\
Enter table size 1-9, 0 to exit: 0
\begin{figure}[h]
\centering
\includegraphics[height=10cm]{1.eps}
\caption{Screenshots of Snake-Like Increment Table}
\end{figure}
\subsection{Specification}

The program should repeatedly request input in the range 0-9 (sample code to get console input is below). Numbers outside this range should be politely rejected. 0 will stop execution of the program. Numbers not parseable as integers should also cause the program to exit.

For input in the 1-9 range, the program should produce a multiplication table as shown above. The formatting does not have to precisely match the above, but the numbers should be in columns. You may use any combination of tabs and spaces for formatting. I would avoid Java's NumberFormatter classes for this one, but if you're feeling daring, go for it.

% If you have more than one objective, uncomment the below:
%\begin{description}
%\item[First Objective] \hfill \\
%Objective 1 text
%\item[Second Objective] \hfill \\
%Objective 2 text
%\end{description}

%----------------------------------------------------------------------------------------
%	SECTION 2
%----------------------------------------------------------------------------------------

\section{Algorithm}

As is shown above, the numbers in odd-numbered lines are increase, while in even-numbered lines, they are decrease. So we can classify this project into two cases. Assume that the first line is numbered as line $i=0$, and a is the order of the table.

\begin{compactitem}
\item For \textbf{odd-numbered} lines��the numbers will range from (a $\times$ i + 1) to (i + 1) $\times$ a.
\item For \textbf{even-numbered} lines, the numbers will range from (i + 1) $\times$ a downto (a $\times$ i + 1).
\end{compactitem}

There are also some details to be considered. For example, the robustness of the input of a. As I use the class Scanner to input from my keyboard, anything that is not integer, such as decimals, letters, will result in an exception, which will cause the process (thread, in Java) to exit. Even if you input an integer, it may be also undesirable. According to the demands, only numbers from 0 to 9 can be used. So we have to add a limit to the input, which is achieved by IF statement.

%----------------------------------------------------------------------------------------
%	SECTION 3
%----------------------------------------------------------------------------------------

\section{Results and Conclusions}
\subsection{Environment}
\begin{compactitem}
\item Windows 8
\item NetBeans IDE 7.3
\end{compactitem}
\subsection{Screenshots of the result}
Use Command Line to compile and execute the program in Figure 1.
\subsection{Thoughts}
This is my first Java program. I find it not hard after I know the differences between Java and C++. I think Java has many things such as different classes that can make program even easier.
%----------------------------------------------------------------------------------------


\end{document}
